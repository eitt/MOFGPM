\documentclass[12pt]{article}
\usepackage[a4paper,margin=1in]{geometry}
\usepackage{setspace}
\usepackage{amsmath,amssymb,mathtools}
\usepackage{booktabs}
\usepackage{newtxtext,newtxmath}
\setstretch{1.15}

\title{Methodology: Baseline-Anchored Sensitivity Analysis with MOFGPM and Jim{\'e}nez Linearization}
\date{}

\begin{document}
\maketitle

This section presents a decision-support pipeline for Emergency Department (ED) resource planning under uncertainty. The methodology is explicitly \emph{baseline-anchored}: it begins from a realistic staffing configuration already used (or constrained) in practice, and then extends the analysis through systematic sensitivity runs to (i) diagnose feasibility of service targets under uncertainty and (ii) map interpretable tradeoffs between service satisfaction and cost. Rather than producing a single ``optimal'' plan, the pipeline produces a feasibility envelope and a cost--service tradeoff surface that managers can use for negotiation and planning.

\section{Problem statement and baseline-anchored design}
Patients arrive over time, are assigned a triage category, and follow triage-dependent activity routes. Activities consume shared resources (Doctor, Nurse, Assistant, Specialist). A staffing plan specifies available capacity per resource type. Given a staffing plan, we construct a feasible schedule that respects patient precedence and resource availability, compute waiting-time performance and staffing cost, and quantify goal satisfaction using max--min Multiobjective Fuzzy Goal Programming (MOFGPM).

The baseline-anchored design directly addresses the managerial gap identified in the Introduction: decision-makers do not start from scratch but from an existing staffing configuration. Consequently, the primary outputs are (a) baseline feasibility and satisfaction, (b) feasibility boundaries under uncertainty, and (c) cost--service tradeoffs for alternative staffing levels.

\section{Case inputs: resources, staffing, and cost}
Staffing is analyzed in two modes: (i) a baseline fixed staffing configuration representing current operations or policy constraints, and (ii) sensitivity sampling within lower/upper bounds to explore alternative capacity levels. Table~\ref{tab:staff} summarizes resource types, unit costs, baseline staffing, and practical bounds used in sensitivity runs.

\begin{table}[h!]
\centering
\begin{tabular}{lcccc}
\toprule
Resource type & Unit cost & Baseline staff & Lower bound & Upper bound \\
\midrule
Doctor & 200 & 3 & 1 & 6 \\
Nurse & 100 & 3 & 1 & 8 \\
Assistant & 50 & 6 & 1 & 10 \\
Specialist & 300 & 1 & 0 & 2 \\
\bottomrule
\end{tabular}
\caption{Resource types, unit costs, baseline staffing, and sensitivity bounds.}
\label{tab:staff}
\end{table}

For a staffing vector $S=\{S_r\}_{r\in\mathcal{R}}$, the total staffing cost is the linear aggregation of unit costs and staffing quantities:
\begin{equation}
\mathrm{Cost}(S) = \sum_{r\in\mathcal{R}} SC_{r}\,S_{r}.
\label{eq:cost}
\end{equation}
Equation~\eqref{eq:cost} links \emph{capacity decisions} to \emph{financial sustainability}: increasing staffing increases the capacity available to meet demand but also increases cost. This cost term later enters MOFGPM through a satisfaction (membership) function that reflects a manager’s budget aspiration and tolerance.

\section{Uncertainty modeling}
\subsection{Patient arrivals by triage (demand model)}
Arrival streams are generated by fitted distributions per triage level, then accumulated into a representative day timeline. Let $TA_i$ denote the arrival time of patient $i$ and $c(i)$ denote the triage class of patient $i$:
\begin{equation}
TA_{i}: \text{arrival time of patient } i, \qquad c(i): \text{triage class of patient } i.
\label{eq:arrivals}
\end{equation}
Equation~\eqref{eq:arrivals} defines the \emph{demand side} of the system. The collection of arrival times $\{TA_i\}$ determines when patients enter the system and therefore how demand overlaps over time. Holding the same realized arrival stream across uncertainty scenarios (when desired) isolates the effect of service-time uncertainty and staffing on feasibility and performance.

\subsection{Service-time uncertainty and fuzzy mapping}
Service durations are uncertain due to heterogeneity in patient condition, operational interruptions, and clinical complexity. To integrate heterogeneous uncertainty descriptions into one framework, we map each activity duration to a fuzzy triple $(p,m,o)$ representing an optimistic value $p$, a central value $m$, and a pessimistic value $o$. This supports scenario-based analysis while maintaining a consistent defuzzification rule.

For triangular activities, we use $(p,m,o)=(L,M,U)$ where $M$ is the most likely duration. For uniform activities over $(a,b)$ we use $(p,m,o)=\left(a,\frac{a+b}{2},b\right)$. For exponential activities, we retain Jim{\'e}nez’s linearization to produce a bounded representation from a location parameter $a$ and scale $\sigma$:
\begin{align}
(p,m,o) &= (L,M,U) && \text{triangular} \label{eq:fuzzy_tri}\\
(p,m,o) &= \left(a,\frac{a+b}{2},b\right) && \text{uniform}(a,b) \label{eq:fuzzy_unif}\\
(p,m,o) &= \left(a,\;a+\frac{\sigma}{2},\;a+\sigma\right) && \text{exponential via Jim{\'e}nez}. \label{eq:fuzzy_jimenez}
\end{align}
Equations~\eqref{eq:fuzzy_tri}--\eqref{eq:fuzzy_jimenez} define how uncertainty in durations is represented. In particular, Equation~\eqref{eq:fuzzy_jimenez} preserves the Jim{\'e}nez interval concept $[a,a+\sigma]$ while defining a central value $a+\sigma/2$ to create a triangular-like fuzzy triple. In implementation, $a$ may be set to the mean or median of the fitted exponential distribution depending on robustness needs.

To evaluate schedules, the pipeline converts the fuzzy triple to a single crisp duration using an expected-value defuzzification operator:
\begin{equation}
EI(\tilde{T}) = \frac{p + 2m + o}{4}.
\label{eq:defuzz}
\end{equation}
Equation~\eqref{eq:defuzz} provides the \emph{expected scenario} duration used in the scheduling module. Conceptually, it balances optimistic and pessimistic extremes while placing higher weight on the central value, ensuring that the schedule evaluation reflects both uncertainty and clinical plausibility.

\section{Scheduling-based evaluation (capacity meets demand)}
Given arrivals (demand) and staffing (capacity), the pipeline constructs a feasible schedule for all required patient activities. Let $ST_{ij}$ and $ET_{ij}$ denote start and end times of activity $j$ for patient $i$. Let $d_{ij}$ be the defuzzified duration, typically $d_{ij}=EI(\tilde{T}_{ij})$ from Equation~\eqref{eq:defuzz}. Precedence and timing are:
\begin{align}
ST_{i1} &\ge TA_{i} \label{eq:prec1}\\
ST_{ij} &\ge ET_{i,j-1}, \quad \forall j>1 \label{eq:prec2}\\
ET_{ij} &= ST_{ij} + d_{ij}. \label{eq:endtime}
\end{align}
Equations~\eqref{eq:prec1}--\eqref{eq:endtime} ensure clinical and temporal feasibility. Equation~\eqref{eq:prec1} prevents service before arrival (no demand before entry). Equation~\eqref{eq:prec2} enforces within-patient precedence (the route must be followed). Equation~\eqref{eq:endtime} links start times to completion times using uncertainty-adjusted durations.

Resource feasibility (the \emph{capacity side}) is enforced through a serial schedule generation scheme: when a task requires one or more resource types, it is assigned to the earliest available units of each required resource type, and its start time is set to the latest of (i) the precedence-implied release time and (ii) the selected resource availability times. This operationally enforces the ``balance between demand and capacity'': if arrivals overlap heavily or staffing is low, resource availability delays start times and increases waiting.

\subsection{Waiting-time metric and triage feasibility}
Waiting time is computed as the time a patient spends waiting between arrival or completion of a prior activity and the start of the next activity:
\begin{align}
WT_{i1} &= ST_{i1} - TA_{i} \label{eq:wt1}\\
WT_{ij} &= ST_{ij} - ET_{i,j-1}, \quad \forall j>1 \label{eq:wt2}\\
\mathrm{Wait}(S) &= \sum_{i}\sum_{j} WT_{ij}. \label{eq:wait_total}
\end{align}
Equations~\eqref{eq:wt1}--\eqref{eq:wait_total} define the service-quality metric used in MOFGPM. The aggregate $\mathrm{Wait}(S)$ is interpretable as total accumulated waiting time across all patients and activities under staffing plan $S$. This quantity increases when demand intensity (arrivals) or service durations exceed available capacity, and it decreases when additional staffing increases capacity.

In addition to aggregate waiting, triage categories impose service-level requirements. Let $WT^{\mathrm{early}}_i$ denote cumulative waiting time over early-care activities (e.g., up to $j\le J_0$ with $J_0=4$ in the implementation), and let $MT_{c}$ denote the triage-specific maximum tolerated threshold. Feasibility requires:
\begin{equation}
WT^{\mathrm{early}}_i \le MT_{c(i)}, \quad \forall i.
\label{eq:triage_sla}
\end{equation}
Equation~\eqref{eq:triage_sla} provides a manager- and clinician-relevant feasibility rule: even if the aggregate waiting time is acceptable on average, excessive early-care delays for high-acuity patients are not permitted. In the computational pipeline, violations of \eqref{eq:triage_sla} are heavily penalized so that solutions are first driven toward feasibility before improving satisfaction.

\section{Multiobjective Fuzzy Goal Programming (MOFGPM)}
Cost and waiting are conflicting objectives: increasing staffing reduces waiting but increases cost. Instead of using fixed weights (which are difficult to justify operationally), MOFGPM converts each objective into a satisfaction degree in $[0,1]$ using piecewise-linear membership functions defined by aspiration levels (Goal) and unacceptable limits (Max).

For cost:
\begin{align}
\mu_{\mathrm{Cost}}(\mathrm{Cost}) &= 1, && \mathrm{Cost} \le G_{\mathrm{Cost}} \label{eq:mu_cost_1}\\
\mu_{\mathrm{Cost}}(\mathrm{Cost}) &= \frac{C_{\max}-\mathrm{Cost}}{C_{\max}-G_{\mathrm{Cost}}}, && G_{\mathrm{Cost}}<\mathrm{Cost}<C_{\max} \label{eq:mu_cost_2}\\
\mu_{\mathrm{Cost}}(\mathrm{Cost}) &= 0, && \mathrm{Cost} \ge C_{\max}. \label{eq:mu_cost_3}
\end{align}
Equations~\eqref{eq:mu_cost_1}--\eqref{eq:mu_cost_3} encode the idea that costs at or below $G_{\mathrm{Cost}}$ fully satisfy the budget goal, costs above $C_{\max}$ are unacceptable, and costs in between reduce satisfaction linearly.

For waiting:
\begin{align}
\mu_{\mathrm{Wait}}(\mathrm{Wait}) &= 1, && \mathrm{Wait} \le G_{\mathrm{Wait}} \label{eq:mu_wait_1}\\
\mu_{\mathrm{Wait}}(\mathrm{Wait}) &= \frac{W_{\max}-\mathrm{Wait}}{W_{\max}-G_{\mathrm{Wait}}}, && G_{\mathrm{Wait}}<\mathrm{Wait}<W_{\max} \label{eq:mu_wait_2}\\
\mu_{\mathrm{Wait}}(\mathrm{Wait}) &= 0, && \mathrm{Wait} \ge W_{\max}. \label{eq:mu_wait_3}
\end{align}
Equations~\eqref{eq:mu_wait_1}--\eqref{eq:mu_wait_3} define a service-quality satisfaction scale: the ED is fully satisfactory if the waiting metric is at or below $G_{\mathrm{Wait}}$, unacceptable if it exceeds $W_{\max}$, and graded linearly in between. Importantly, in sensitivity studies $W_{\max}$ can be varied to trace feasibility boundaries and to show how ``tight'' waiting tolerances restrict feasible operation under uncertainty.

The max--min MOFGPM objective then maximizes the minimum satisfaction level $\lambda$ across objectives:
\begin{align}
\max \ & \lambda \label{eq:mofgpm_obj}\\
\text{s.t. } & \lambda \le \mu_{\mathrm{Cost}}, \quad \lambda \le \mu_{\mathrm{Wait}} \label{eq:mofgpm_cons}\\
& 0 \le \lambda \le 1. \label{eq:mofgpm_bounds}
\end{align}
Equations~\eqref{eq:mofgpm_obj}--\eqref{eq:mofgpm_bounds} enforce a \emph{balanced} tradeoff: $\lambda$ improves only if both cost and waiting satisfaction improve, since $\lambda$ is bounded by the smaller membership. This directly supports the managerial goal of avoiding solutions that are excellent on one dimension but unacceptable on the other.

\section{Solution approach: baseline and sensitivity pipeline}
The optimization stage uses a genetic algorithm (GA) that searches over task-priority permutations and decodes each chromosome into a feasible schedule under precedence and resource feasibility rules. The GA fitness is defined by the MOFGPM objective: for each decoded schedule, we compute $\mathrm{Cost}(S)$ via Equation~\eqref{eq:cost}, $\mathrm{Wait}(S)$ via Equation~\eqref{eq:wait_total}, memberships via Equations~\eqref{eq:mu_cost_1}--\eqref{eq:mu_wait_3}, and $\lambda$ via Equations~\eqref{eq:mofgpm_obj}--\eqref{eq:mofgpm_bounds}. Schedules violating the triage feasibility rule in Equation~\eqref{eq:triage_sla} receive strong penalties.

Two execution modes are used:
(i) \textbf{baseline case} with fixed staffing to quantify current feasibility and satisfaction, and
(ii) \textbf{sensitivity analysis} by sampling staffing vectors within practical bounds (Table~\ref{tab:staff}) and/or varying $W_{\max}$ to map feasibility boundaries and cost--service tradeoffs under optimistic/expected/pessimistic duration scenarios.

\section{Feasibility boundaries and tradeoff reporting}
Each tested staffing plan reports $\mathrm{Cost}$, $\mathrm{Wait}$, $\mu_{\mathrm{Cost}}$, $\mu_{\mathrm{Wait}}$, $\lambda$, and a feasibility flag determined primarily by Equation~\eqref{eq:triage_sla} and by whether both memberships are positive (i.e., performance remains within tolerance limits). Aggregating these outputs produces (a) feasible/infeasible regions and (b) an achievable tradeoff frontier between cost and service satisfaction.

\section{Validation and robustness checks}
Robustness is evaluated through repeated stochastic replications and confidence-interval summaries of waiting-time and triage-level service metrics. This step ensures that baseline conclusions and sensitivity trends are not artifacts of a single simulated day.

\subsection{External validation with FlexSim}
In addition to the methodology runs (baseline and sensitivity), a FlexSim discrete-event simulation can be used to validate the GA-based approach. Baseline staffing and selected sensitivity staffing samples are replicated in FlexSim using the same triage pathways, arrival logic, and service assumptions. Validation compares key outputs between GA+MOFGPM and FlexSim (e.g., total waiting time, triage-level service attainment, and resource utilization). Close agreement supports the validity of the optimization pipeline and its managerial interpretations.

\end{document}
